\documentclass{article}
\title{Math 120 Assignment 1}
\author{Isaac Winger 20167000}
\date{}
\pagenumbering{gobble}

\usepackage{graphicx}
\usepackage{amsfonts}
\graphicspath{ {./} }

\begin{document}
    \begin{enumerate}
        \item \begin{enumerate} \item We start with the definition of a rational number. As in equation one it must be a fraction of two integers where the denominator is not zero.
            \begin{equation} \frac{p}{q} \mid p, q \in \mathbb{Z}, q \neq 0 \end{equation}
                      Now we can define the problem. A and b are defined as some rational number, and c is the sum of both numbers.
                      \begin{equation} a + b = c \end{equation}
                      Since a and b are both rational numbers, we can replace them with the definition of a rational number as seen in equation one.
                      \begin{equation} \frac{p}{q} + \frac{m}{n} = c \end{equation}
                      From this equation, both fractions can be added together to simplify the value of c.
                      \begin{equation} \frac{pn}{qn} + \frac{mq}{qn} = c \end{equation}
                      \begin{equation} \frac{pn + mq}{qn} = c \end{equation}
                      This shows that c must also be a rational number. It is known that neither q nor n are zero because they couldn't be zero by definition of a rational number, so the denominator is not zero. Both the numerator and denominator are integers since adding two integers always results in another integer, and integer multiplication can also be written in the form below, where the sign of the sum is flipped if y is negative.
                      \begin{equation} x \cdot y = (x + x + x \ldots + x)\end{equation}
                      Therefore, by definition of a rational number from equation one, any rational number a and b added together will result in another rational number.

                \item Similar to the previous question, the problem can be represented using the definition seen in equation one. Numbers a and b are rational numbers, and c represents the sum.
                      \begin{equation} a \cdot b = c \end{equation}
                      \begin{equation} \frac{p}{q} \cdot \frac{m}{n} = c \end{equation}
                      Both of these fractions can be multiplied together to get a simplified value for c.
                      \begin{equation} \frac{pm}{qn} = c \end{equation}
                      Once again this final fraction follows the definition of a rational number. Neither q nor n were initially zero, so it follows that the product of q and n cannot be zero. Both the numerator and denominator are defined by integer multiplication, so they must both be integers. Therefore any rational number a and b multiplied together will result in another rational number.

                \item We first define x as the initial number to be squared. The exponent $x^{4}$ can be broken down to give a more useful form.
                      \begin{equation} x^{4} = (x^{2})^{2} \end{equation}
                      $x^{2}$ is known to be rational, so $x^{2}$ can be replaced with the definition of the rational number from equation one.
                      \begin{equation} x^{4} = (\frac{p}{q})^{2} \end{equation}
                      Since the square of a number is equivilant to multiplying a number by itself once, $x^{4}$ is equivilant to the following product.
                      \begin{equation} x^{4} = \frac{p}{q} \cdot \frac{p}{q} \end{equation}
                      This is just rational number multiplication. From question b it is known that $x^{4}$ must be rational as well. Therefore for any number x, if $x^{2}$ is rational, then $x^{4}$ must also be rational.
        \end{enumerate}
    \end{enumerate}
    \end{document}
