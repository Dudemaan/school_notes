\documentclass{article}
\pagenumbering{gobble}
\usepackage{amsfonts}

\begin{document}
    \begin{enumerate}
            \item \begin{enumerate}
                    \item To get the natural domain of this function we must first break it into three functions: $f(x)=\sqrt{1-x}$, $g(x)=\sqrt{x}$, and $h(x)=25-x^{2}$ where $f \circ g \circ h$ is the full function. To get the natural domain we must determine what . With square root functions, the input must be $\geq 0$ to remain in $\mathbb{R}$. For $f(x)$ this is the case when $x \leq 1$. So the codomain of $g(x)$ must be a subset of $\lbrack 1,-\infty)$. Since $g(x)$ cannot output a negative number, the codomain now becomes $\lbrack 1,0 \rbrack$. This means that the domain for $g(x)$ must be $\lbrack 1,0 \rbrack$. Finally, since the codomain for $h(x)$ is $\lbrack 1,0 \rbrack$ we can determine that the only inputs that result in that output are between $\lbrack 2\sqrt{6}, 5 \rbrack$ and $\lbrack -5, -2\sqrt{6} \rbrack$. So the natural domain of this function is:
                          \[\lbrack -5, -2\sqrt{6} \rbrack \cup \lbrack 2\sqrt{6}, 5 \rbrack\]
                    \item
            \end{enumerate}
    \end{enumerate}
\end{document}
